\section{Preliminary Research in 2D}
\markboth{Preliminary Research in 2D}{}
\label{sec:preliminary}
When we first tackled the problem, we started with 2D setting.
In figure~\ref{fig:mvcInput} we see a blob and some lines passing through it.
The user supposedly marked the parts of the lines that are inside the blob, although actually it was done automatically.
Then, the algorithm computed the interpolated 2D blob induced by this input.


\subsection{MVC Style Segmentation}

Our first attempt at constructing an interpolation was in the spirit of MVC \cite{Floater200319}.
We computed for each point in the image the aggregated influence of the input segments.
Each segment is labeled ``inside'' or ``outside''. 
Its impact is inverse-proportional to the integral of the distances from the target point to the segment points:
\[
\int_a^b \mathrm{\frac{1}{\|x-c\|}} \,\mathrm{d}x
\]
In order to label a target point, we simply sum-up all the impacts with their proper sign (1 for inside, -1 for outside) and the result sign is the label for the target point.
Figure~\ref{fig:mvcNoDT} shows the result of this method.

A further improvement was achieved by using sort of ``distance transform'' weight to outside segment impact.
In Figure~\ref{fig:mvcIntegral} we see a segment $ab$, labeled ``outside'', and a random point $c$.
We want to calculate the impact of segment $ab$ on the labeling of point $c$.
We found out that it was better to let the distance of the segment points from the blob affect the order of impact on the target point:
\[
\int_a^b \mathrm{\frac{x}{\|x-c\|}} \,\mathrm{d}x 
\]
The result is shown in figure~\ref{fig:mvcDT}.

\begin{figure}[ht]
\centering
  \includegraphics[width=1.0\textwidth]{images/draw1_integral.png}
  \caption[``MVC'' Calculation]{
  ``MVC'' Calculation.  The segment $ab$ affects the label of $c$.
  }\label{fig:mvcIntegral}
\end{figure}


\clearpage

\iffalse

\begin{figure}[ht]
\centering
 \begin{minipage}[h]{.35\textheight}
  \centering
  \includegraphics[width=.35\textheight]{images/draw1_input.png}
  \vspace{.1cm}
  \end{minipage}
 \begin{minipage}[h]{.35\textheight}
  \centering
  \includegraphics[width=.35\textheight]{images/draw1_no_dt.png}
  \vspace{.1cm}
  \end{minipage}
 \begin{minipage}[h]{.35\textheight}
  \centering
  \includegraphics[width=.35\textheight]{images/draw1_dt.png}
  \end{minipage}
  \caption[``MVC'']{
  ``MVC'' (a) Input. (b) Result w/o DT. (c) Result with DT.
  }\label{fig:mvc}
\end{figure}

\fi

\begin{figure}[t]
\centering
  \includegraphics[width=.3\textheight]{images/draw1_input.png}
  \caption[``MVC'' Input]{
  ``MVC'' Input.
  }\label{fig:mvcInput}
\end{figure}

\begin{figure}[t]
\centering
  \includegraphics[width=.3\textheight]{images/draw1_no_dt.png}
  \caption[``MVC'' w/o DT]{
  ``MVC'' without using distance transform
  }\label{fig:mvcNoDT}
\end{figure}

\begin{figure}[b]
\centering
  \includegraphics[width=.3\textheight]{images/draw1_dt.png}
  \caption[``MVC'' with DT]{
  ``MVC'' using distance transform
  }\label{fig:mvcDT}
\end{figure}

\clearpage

\subsection{RBF Segmentation}\label{subsec:rbfSeg}
Another interpolation method that we tried is RBF i.e., radial basis functions approximation.
A radial basis function (RBF) is a real-valued function whose value depends only on the distance from the origin, so that $\phi(\mathbf{x}) = \phi(\|\mathbf{x}\|)$; or alternatively on the distance from some other point c, called a center, so that $\phi(\mathbf{x}, \mathbf{c}) = \phi(\|\mathbf{x}-\mathbf{c}\|)$. Any function $\phi$ that satisfies the property $\phi(\mathbf{x}) = \phi(\|\mathbf{x}\|)$ is a radial function. The norm is usually Euclidean distance, although other distance functions are also possible.
For example, in figure~\ref{fig:rbfDemo} we see how 3 Gaussian RBF functions produce a smooth function.
Given spatial locations (``centers'') $x_i$ for $i=1,\ldots,n$, each associated with some scalar value $f_i$. The value at an arbitrary spatial location $x$ can be computed by
\[
f(x) = \sum_{i=1}^{n} w_i \phi (\|x - x_i\|)
\]
where $\phi (r)  = r^{2}  \ln (r)$ is a radial basis kernel of type ``thin plate spline''. The weights $w_i$ are computed to satisfy the interpolation property, namely $f(x_i)=f_i$ for all $i$. The computation of $w_i$ involves solving a linear system with $n$ equations, after which the evaluation of $f(x)$ for any $x$ involves only a simple summation as in the above formula. Such interpolation $f$ is smooth in the sense that it has the minimal integral of squared second derivatives, the so-called thin-plate spline energy.

To apply RBF to our problem, we create a signed area via interpolation so that the zero isoline (iso-contour) of the area interpolates the boundary points. We start by creating for each in-out crossing, two RBF centers along the normal of the image gradient, one on the inside and the other on the outside. The inside center is given a value of 1 and the outside center has value -1. The signed area is then computed using RBF from the values at these centers.

\begin{figure}[htb]
	\centering
		\includegraphics[width=1.0\textwidth]{images/rbf_1D_demo.png}
			\caption[1D RBF Demonstration]{
				1D RBF Demonstration.
				The dashed functions are Gaussian RBFs;
				The solid function is their weighted sum.
		}\label{fig:rbfDemo}
\end{figure}

In figure~\ref{fig:rbfNoGrad} we didn't use the image gradient, but positioned the inside/outside RBF centers along the segment itself, around the in-out crossing.
Figure~\ref{fig:rbfGrad} shows an improved result when using the image gradient for positioning the RBF centers.
Another idea for improvement was to allow the contour ``breath'' along the gradient-normal, searching for the patch with the most similar signature to the nearest in/out crossing patch signature.
The result of this idea is illustrated in figure~\ref{fig:rbfBreath}.
In this figure, the circles' radius indicates the degree of the aforementioned similarity.

\begin{figure}[htb]
	\centering
		\includegraphics[width=1.0\textwidth]{images/draw3_rbf_ignore_gradient.png}
			\caption[2D RBF w/o Using Image Gradient]{
				2D RBF w/o Using Image Gradient.
		}\label{fig:rbfNoGrad}
\end{figure}

\begin{figure}[htb]
	\centering
		\includegraphics[width=1.0\textwidth]{images/draw3_rbf_gradient.png}
			\caption[2D RBF Utilizing Image Gradient]{
				2D RBF Utilizing Image Gradient.
		}\label{fig:rbfGrad}
\end{figure}

\begin{figure}[htb]
	\centering
		\includegraphics[width=1.0\textwidth]{images/draw3_rbf_breath.png}
			\caption[2D RBF With Contour ``Breathing'']{
				2D RBF With Contour ``Breathing''.
		}\label{fig:rbfBreath}
\end{figure}

