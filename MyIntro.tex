% (where we describe the motivation for the work and how we propose to handle it)

\section{Introduction}
\markboth{Introduction}{}
\label{sec:intro}

Medical imaging is increasingly expanding its application and contribution to clinical procedures. It has been more than 100 years ago, when X-Ray was first shown to enable non-invasive visualization of tissue. Since then medical imaging has developed to become an essential tool of any clinical intervention. The ability to visualize body organs and tissue without opening the body to do so has become a common practice. With progress in technology medical imaging is used not only to diagnose but also to perform image guided therapeutic and surgical interventions. In many cases the therapy itself is also non-invasive (e.g. radiation) which makes the medical imaging the sole information by which the surgery is being planned and performed.

Typically, a specific volume in the anatomy needs to be targeted (e.g. tumor) whereas adjacent structures need to be spared and unaffected. The distinction of these different volumes is within the expertise of the treatment planning physician. Deep understanding of the anatomy, as well as the specific disorder is required to adequately define the different volume regions. The treatment volumes are commonly not easily delineated by a specific visible boundary. The task of outlining the treatment boundary is more detailed than actually segmenting the volume into separate structures. Whereas segmenting a structure may rely on separating surfaces, different gradients of properties of specific sub volumes, delineating a boundary for clinical treatments is based on clinical know-how and thus necessarily require manual expert interaction.

As a result, the planning of an image guided therapy requires multiple volumes to be delineated with the aid of physician guidance with great accuracy and resolution. This interactive delineating process becomes a challenging task. Despite progress in 3D visualization in many areas medical imaging is still strictly analyzed in 2D planar slices of the 3D anatomical data set. The marking of the volume boundary is thus achieved by describing its contour on 2D planar slices in an iterative manner. Many segmentation and outlining tools can aid the user in delineating a contour in 2D. Outlining in 2D can be effective as the user can perceive the entire boundary at onces. However, this cannot be easily scalable to 3D volumetric delineation. In 3D there is no simple way to visualize the entire shape and there is no single view that can guide the user as to where an additional marking is needed. In complex situations this is both tedious and there is no clear way for the user to identify the most efficient way to mark the boundary. Thus, the delineation of the 3D target shape is applied on a series of 2D slices. 

In this paper we present a scheme to minimize the amount of user iterations that is required to delineate the volume boundary. The key idea is to let the system analyze the data and the progressive information provided by the user and suggest the user the next slice to be delineated. This active system avoids the user the need to review multiple slices to decide where additional input may be most helpful. A careful analysis the volume and its geometry lead to a more efficient interactive procedure to reach the target boundary while minimizing the manual effort required from the user.

The problem of ``where to delineate next'' is an ill-posed geometric problem since the target shape cannot be defined without the expert know-how. Our approach is to iteratively reconstruct a 3D shape from the series of 2D outlines provided so far by the user, analyze the volume in the vicinity of the reconstructed surface and to use a heuristic to estimate the local confidence that the surface is on the target boundary. The next slice prompted to the user is the one for which the confidence is the lowest. 
We show that with our active system requires less manual labor by the expert.

%Well, we count the number of slice that are needed, but we didn't count the effort of searching for the next slice... here the system prompt the next slices!
%I want to display a 10x4 or even 20x5 array of slices that represent the display that the expert sees. It will help explaining that the user needs to search for the next slice... here we save this time...
