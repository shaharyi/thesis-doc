% where we describe previous work and references to the basic modules by which we will build our approach

\section{Related Work}
\markboth{Related Work}{}
\label{sec:related}

Delineation of anatomical structures is closely related to a more generic, long researched topic of segmentation of 3D data set. Automatic segmentation of medical images is a difficult task since medical images are complex in nature and rarely have any simple linear feature \cite{sharma2010}. Segmentation has been explored to be based on gray level features \cite{487786} or texture based features \cite{Wang1996509, Sharma2008}.   However, automatic image segmentation is particularly difficult due to restrictions imposed by image acquisition, pathology and biological variation and operator interaction is still required for error correction in the event of an inadequate result \cite{Olabarriaga2001127}. 

Common techniques use the general location of the structure from some user defined initialization and then rely on a model or on high gradient to control and perform the segmentation \cite{Kass88snakes:active, Criminisi2008, Li:2004:LS:1186562.1015719, Grady:2006:RWI:1175896.1176220}.  Statistical analysis has been suggested to distinguish between inner and outer regions \cite{902291, 4337766}.  Adaptive methods suggested using the characteristics of the boundary from user input and active contour evolves to estimate the boundary \cite{Gao20121216}.

The delineation process of a volume in a medical imaging is a key element in image guided procedures which are based on digital medical imaging. The vast amount of data, details and know-how, required make it tedious and time consuming \cite{Li:2004:LS:1186562.1015719}. Semi-automated techniques have been trying to improve effectiveness while taking advantage of the physician knowhow to overcome the anatomical complexity. 3D-Live wire \cite{Poon2008}, is a technique that extends the 2D live wire \cite{Barrett97interactivelive-wire}. It is based on based contours generated semi-automatically on a selected set of slices are used as seed points on new unseen slices in different orientations. The seed points are calculated from intersections of user-based live-wire techniques with new slices. The technique reduces significantly the number of slices the user has to delineate, however the result strongly depends on the use selecting the best slices to achieve an effective delineation \cite{Poon2008}.  Similar interactive  Live Wire techniques \cite{Barrett97interactivelive-wire} have been shown to enable extracting a contour in one-fifth of the time required for manual tracing, and with 4.4 times greater accuracy and 4.8 times greater reproducibility.  A promising statistical analysis with user interaction, \cite{hu:4547} supports user strokes on the images to identify the characteristics of the organ to be delineated and its surroundings. Based on user input from sample slices the boundary is estimated on nearby slices.

The approach we choose to handle the 3D delineation goes beyond the various segmentation techniques. On top of finding the most effective way to estimate a boundary based on a given user input, there is a need to guide the user so that his next input will contribute the most. This is a non trivial task in a 3D environment. The common interface for anatomical structures is a set of planar images that slice through the analyzed structure. In a complex enviroment the interaction is based on viewing mutiple (e.g. doezens) of images in vairous orientations\cite{Top2011} . The delineation process is broken to two speciffic tasks, the selection of a particular image for interaction and then visualizing and editing the contour on the selected image. Selecting the most effective image to delineate is a challanging task which we focus on in this work.

This problem is related to research done in 3D object recognition in robotics, rapid prototyping and more. A more generic description of the problem may be defined as finding the Next Best View on a volumetric data set \cite{1087372} . Next Best View approaches can be classified as surface, volumetric or global based on the application and the domain of their focus.  Typically surface algorithm, aims at finding areas (arcs) where the surface of the object is occluded as candidates for the Next Best View \cite{MaverB93} selecting the one with the maximal arc. Volumetric approaches label voxels in a 3D space as known or unknown and will seek for the next camera pose that will have the greatest reduction in unknown voxels \cite{Banta95a}. Similarly different Information Gain Functions can be defined and the Next Best Scan is selected to be the path in which the largest Information Gain which reflects the amount of unknown voxels that will be viewed in the scan \cite{KriegelRBNSH12, Wong99nextbest}.

In the context of medical imaging the ``spotlight'' approach, \cite{Top2011} suggest a method that highlights to the user an image plane where additional input or editing will be most effective. Initially the user manually selects a plane and describes the structure boundary on that plane. In an iterative process the user edits the boundary on 2D slices to improve the overall outcome. This work introduces an Active Learning technique in which the system suggests the most effective next slice for delineation. The 3D space is sampled and voxels are graded by the uncertainty in their labeling. The effective plane is suggested as the one that intersects with the voxels where uncertainty is the highest. The user then provides labeling for the multiple points which is considered as a batch query for the active learning process. The uncertainty invovles grading, for the sampled voxels, by defining an uncertainty field which invovles entropy, boundary energy, regional energy and smoothness energy. Since grading the entire pixel uncertainties and select the one with the most uncertainty as the most effective. However this type of selection does not garuantee the most globally effective plane but rather the best sample.
