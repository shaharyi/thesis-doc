% Academic abstract structure from Wikipedia:
%    The research focus (i.e. statement of the problem(s)/research issue(s) addressed);
%    The research methods used (experimental research, case studies, questionnaires, etc.);
%    The results/findings of the research; and
%    The main conclusions and recommendations

%--------------------------------------------------------------------------
%% Abstract section.
\section*{Abstract}
\markboth{Abstract}{}
\label{sec:abstract}
% short intro to the problem space and the opportunity

%avoid warning "`destination with the same identifier (nam e{page.1}) has been already used, duplicate ignored"'
\hypersetup{pageanchor=false}

%\subsection*{Research Focus}
In this thesis we present a scheme to reduce the amount of user iterations required to segment an object by delineating on cross-section planes.
Starting with an initial segmentation created from a small number of delineated curves, the algorithm progressively analyzes the uncertainty of segmentation with respect to the image features and suggests the ``next plane'' for delineation that would maximally resolve the uncertain regions.
The problem of ``where to delineate next'' is an ill-posed geometric problem since the target shape cannot be defined without the expert know-how. Compared with the few prior art on this problem, we adopt a simpler computational framework that is made up of an RBF-based curve interpolation method, a distance-based uncertainty metric, and a plane determination approach using density-based clustering.

%\subsection*{Research Methods Used}
We developed an interactive graphic application for this research, with both 2D and 3D interfaces.
This way we could demonstrate the algorithm using both synthetic and real examples and verify that it can be used interactively by timing its operation.
A comparison of this algorithm with random plane-selection algorithm was performed, and it proved to be sufficient, as the random algorithm had pretty good performance on its own, surprizingly enough.

%\subsection*{Results}
The performance of an algorithm in our context is expressed in the graph of ``similarity to ground-truth'' as function of ``number of slices''. 
We compare our algorithm to the stat-of-the-art algorithm on similar data and show that they exhibit similar performance.
It is interesting to note that that all three algorithms (ours, state-of-the-art, and the random) are similar in their performance.
Thus, showing that we beat the random algorithm and compete well with the state-of-the-are, is adequate.
We show that our method uses less than $50\%$ of the number planes than a random selection scheme to achieve $90\%$ segmentation accuracy.

%\subsection*{Conclusions and Recommendations}
We have demonstrated an elegant and efficient method for interactively delineating 3D shapes in a volume.
The RBF interpolation method can be further optimized to be faster and more robust to user inaccuracies. That is, it has to overcome contradicting contours when the difference stems merely from inaccurate user marks.
The cluster analysis of the low-certainty areas can be further developed and fine-tuned. For example, if the first cluster is over twice the second cluster, we may still use the second cluster's PCA combined with the first cluster's PCA in clever ways, e.g., compute the combined weighted PCA, maybe only considering nearly-aligned components of the two clusters.


\hypersetup{pageanchor=true}

\iffalse
In this thesis we present a scheme to reduce the amount of user iterations required to segment an object by delineating on cross-section planes. Starting with an initial segmentation created from a small number of delineated curves, the algorithm progressively analyzes the uncertainty of segmentation with respect to the image features and suggests the ``next plane'' for delineation that would maximally resolve the uncertain regions. The problem of ``where to delineate next'' is an ill-posed geometric problem since the target shape cannot be defined without the expert know-how. Compared with the few prior art on this problem, we adopt a simpler computational framework that is made up of an RBF-based curve interpolation method, a distance-based uncertainty metric, and a plane determination approach using density-based clustering. We demonstrate using both synthetic and real examples that our method uses less than $50\%$ of the number planes than a random selection scheme to achieve $90\%$ segmentation accuracy.
\fi

\iffalse
%Danny's abstract for EG14 (didn't submit eventually)
In this paper we present a scheme to minimize the amount of user iterations that is required to delineate the volume boundary. The key idea is to let the system analyze the data and the progressive information provided by the user and suggest the user the next cross-section to be delineated. The problem of ``where to delineate next'' is an ill-posed geometric problem since the target shape cannot be defined without the expert know-how. Our approach is to iteratively reconstruct a 3D shape from the series of 2D outlines provided so far by the user, analyze the volume in the vicinity of the reconstructed surface and to use a heuristic to estimate the local confidence that the surface is on the target boundary. The next cross-section prompted to the user is the one for which the confidence is the lowest. We show that with our active system requires less manual labor by the expert.

%Tau Ju abstract for ISBI14
In this paper we present a scheme to reduce the amount of user iterations required to segment an object by delineating on cross-section planes. Starting with an initial segmentation created from a small number of delineated curves, the algorithm progressively analyzes the uncertainty of segmentation with respect to the image features and suggests the ``next plane'' for delineation that would maximally resolve the uncertain regions. Compared with the few prior art on this problem, we adopt a simpler computational framework that is made up of an RBF-based curve interpolation method, a distance-based uncertainty metric, and a plane determination approach using density-based clustering. We demonstrate using both synthetic and real examples that our method uses less than $50\%$ of the number planes than a random selection scheme to achieve $90\%$ segmentation accuracy.
\fi
